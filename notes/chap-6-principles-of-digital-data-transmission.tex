\documentclass{article}
\usepackage{graphicx}
\usepackage{amsmath}
\usepackage{amsfonts}
\renewcommand{\baselinestretch}{1}
\setlength{\textheight}{9in}
\setlength{\textwidth}{6.5in}
\setlength{\headheight}{0in}
\setlength{\headsep}{0in}
\setlength{\topmargin}{0in}
\setlength{\oddsidemargin}{0in}
\setlength{\evensidemargin}{0in}
\setlength{\parindent}{.3in}
\graphicspath{{images/}}


\title{Communication Systems \\
    \large Chapter 6 Principles of Digital Data Transmission }
\author{Hunter Mills}
\date{\today}

\begin{document}
    \maketitle

    \medskip
        
    \section{Baseband Line Coding}
    Digital data can be transmitted with multiple different line codes (such as on/off, polar, bipolar). They each have their own
    advantages but the properties a line code should have is
    \begin{itemize}
        \item \textbf{Low Bandwidth}    - Transmission bandwidth should be as low as possible.
        \item \textbf{Power Efficiency} - For a given bandwidth and specified detection error rate, the power should be as low as possible.
        \item \textbf{Error Detection and Correction} - Ability to detect and correct errors (CH13).
        \item \textbf{Favorable Spectral Density} - It is desireable to have 0 PSD at DC.
        \item \textbf{Adequate Timing Content} - It should be able to extract timing/clock information.
        \item \textbf{Transparency} - It should be possible to correctly receive any pattern of 0, 1's.
   \end{itemize}
        
    \subsection{PSD of Various Baseband Line Codes}
    In this section $p(t)$ is a generic pulse with Fourier Transform of $P(f)$. The line code symbol at time $k$ is $a_k$
    and $T_b = 1/R_b$ (bit rate).
    \begin{equation}
        y(t) = \sum a_kp(t-kT_b)
    \end{equation}
    A way to describe the PSD of $y(t)$, not depending on $p(t)$ is to select a PAM signal $x(t)$ that uses unit impulses for the 
    basic then pass it through a filter with impulse response $h(t) = p(t)$. The PSD will be
    \begin{equation}
        S_y(f) = |P(f)|^2S_x(f)
    \end{equation}

    \begin{figure}[h]
        \centering
        \includegraphics[width=0.75\textwidth]{psdpulse}
        \caption{PSD of $p(t)$}
    \end{figure}
    A very good derivation on page 372 of the autocorrelation $\mathcal{R}(\tau)$ where 
    \begin{equation}
        \mathcal{R}(\tau) = \lim_{T \rightarrow \infty}\frac{1}{T}\int_{-T/2}^{T/2}x(t)x(t-\tau)dt
    \end{equation}
    \begin{equation}
        \mathcal{R}(\tau) = \frac{1}{T_b}\sum_n R_n \delta(\tau-nT_b)
    \end{equation}
    where ($|\overline{x}|$ is the time average)
    \begin{equation}
        R_n = \lim_{N \rightarrow \infty} \frac{T_b}{T} \sum_k a_ka_{k+n} = |\overline{a_ka_{k+n}}|
    \end{equation}
    \begin{equation}
        R_0 = \lim_{N\rightarrow \infty} \frac{1}{N}\sum_k a_k^2 = |\overline{a_k^2}|
    \end{equation}
    The PSD $S_x(f)$ is the fourier transform of $\mathcal{R}(\tau)$,
    \begin{equation}
        S_x(f) = \frac{1}{T_b}\sum_n R_ne^{-jn2\pi fT_b}
    \end{equation}
    Since $\mathcal{R}(\tau)$ is an even function of $\tau$
    \begin{equation}
        S_x(f) = \frac{1}{T_b}[R_o + 2 \sum_{n=1}^{\infty}R_n \cos n2\pi fT_b]
    \end{equation}
    and finally
    \begin{equation}
        S_y(f) = |P(f)|^2S_x(f) 
    \end{equation}
    \begin{equation}
        S_y(f) = \frac{|P(f)|^2}{T_b}[\sum_n R_n e^{-jn2\pi fT_b}]
    \end{equation}
    \begin{equation}
        S_y(f) =  \frac{|P(f)|^2}{T_b}[R_0 + 2 \sum_{n=1}^{\infty}R_n \cos n2\pi fT_b]
    \end{equation}

    \subsection{Polar Signaling}
    In polar signaling, a binary $1$ is sent as $p(t)$ and a binary $0$ is sent as $-p(t)$. In this case
    $a_k$ is equally likely to be $1$  or $-1$ and $a_k^2 = 1$.
    \begin{equation}
        R_0 = \lim_{N\rightarrow \infty} \frac{1}{N}\sum_k a_k^2 =  \lim_{N\rightarrow \infty} \frac{1}{N}(N) = 1
    \end{equation}
    Equation 5 shows that $R_n$ is the time average of $a_ka_{k+n}$ and since $a_k$ is equally likely
    to be $1$ or $-1$,
    \begin{equation}
        R_1 = \lim_{N \rightarrow \infty} \frac{1}{N}[\frac{N}{2}(1) + \frac{N}{2}(-1)] = 0
    \end{equation}
    and therefore 
    \begin{equation}
        R_n = 0 \quad n > 0
    \end{equation}
    \begin{equation}
        S_y(f) = \frac{|P(f)|^2}{T_b}R_0 = \frac{|P(f)|^2}{T_b}R_0
    \end{equation}

    \subsection{Constructing a DC Null in PSD by Pulse Shaping}
    Since $S_y(f)$ has a multiplicative factor of $|P(f)|^2$ we can force the PSD to have a DC null 
    such that $p(t)$ doenst have a DC offset.
    \begin{equation}
        P(0) = \int_{-\infty}^{\infty}p(t)dt
    \end{equation}
    If the area under $p(t)$ is zero then there is a DC null.

    \section{Digital Carrier Systems}
    \subsection{Basic Binary Carrier Modulations}
    \subsubsection{ASK}
    ASK is known as \textbf{Amplitude Shift Keying}. The on-off baseband signal $m(t)$ is ($a_k = 0,1$)
    \begin{equation}
        m(t) = \sum_k a_k p(t-kTb) \quad \textrm{where} \quad p(t) = \Pi (\frac{t-T_b/2}{T_b})
    \end{equation}
    The ASK signal is
    \begin{equation}
        \varphi_{ASK}(t) = m(t)\cos \omega_ct
    \end{equation}

    \begin{figure}[h]
        \centering
        \includegraphics[width=6cm, height=7cm]{ask}
        \caption{On-off keyed $m(t)$ modulated by ASK}
    \end{figure}

    \subsubsection{PSK}
    PSK is known as \textbf{Phase Shift Keying}. If a baseband signal $m(t)$ is polar, transmitting a $1$ 
    would produce a pulse $p(t)\cos \omega_ct$ and transmitting a $0$ would produce $-p(t)\cos \omega_ct$.
    This can be rewritten as 
    \begin{equation}
        -p(t)\cos \omega_ct = p(t)\cos (\omega_ct + \pi)
    \end{equation}
    which is a signal that is $\pi$ radians (180deg) apart in phase.
    So when the line code is polar ($a_k = -1,1$), then
    \begin{equation}
        \varphi_{PSK}(t) - m(t)\cos \omega_c t \quad\quad m(t) = \sum_k a_kp(t-kT_b)
    \end{equation}
     
    \subsubsection{FSK}
    FSK is known as \textbf{Frequency shift keying}. This is done by varying the instantaneous frequency,
    where a $0$ is transmitted with $\omega_{c_0}$ and a $1$ is transmitted with $\omega_{c_1}$. 
    FSK can be viewed as a sum of two alternating ASK signals, one with carrier $\omega_{c_0}$ and the other 
    with carrier $\omega_{c_1}$. The binary ASK expression to write a FSK signal is
    \begin{equation}
        \varphi_{FSK}(t) = \sum_k a_kp(t-kT_b)\cos\omega_{c_1}t + \sum_k(1-a_k)p(t-kT_b)\cos\omega_{c_0}t
    \end{equation}
    where $a_k = 0,1$ is on off keying. The FSK is the superposition of two ASK signals with different 
    frequencies but complementary amplitudes. 

    \begin{figure}[h]
        \centering
        \includegraphics[width=6cm, height=7cm]{pskfsk}
        \caption{Polar $m(t)$ modulated with PSK then FSK}
    \end{figure}

    \subsection{PSD of Digital Carrier Modulation}
    In the current section it was shown that ASK, PSK and FSK can be written in the format of 
    $m(t)\cos\omega_ct$. 
    \begin{equation}
        \varphi(t) = m(t)\cos\omega_ct
    \end{equation}
    The PSD of $\varphi(t)$ is
    \begin{equation}
        S_{\varphi}(f) = \lim_{T \rightarrow \infty} \frac{|\Psi_T (f)|^2}{T}
    \end{equation}
    where $\Psi_T (f)$ if the Fourier transform of the truncated signal
    \begin{equation}
        \varphi_T(t) = \varphi(t)[u(t+T/2) - u(t-T/2)]
    \end{equation}
    \begin{equation}
        \varphi_T(t) = m(t)[u(t+T/2) - u(t-T/2)]\cos\omega_ct
    \end{equation}
    \begin{equation}
        \varphi_T(t) = m_T(t)cos\omega_ct
    \end{equation}
    The PSD of $m_T(t)$ is
    \begin{equation}
        S_m(f) = \lim_{T \rightarrow \infty} \frac{|M_T(f)|^2}{T}
    \end{equation}
    and then applying the frequency shift from the $\cos$ term,
    \begin{equation}
        \Psi_T(f) = \frac{1}{2}[M_T(f-f_c) + M_T(f+f_c)]
    \end{equation}
    so finally the PSD of the modulated signal is
    \begin{equation}
        S_{\varphi}(f) = \lim_{T\rightarrow\infty} \frac{1}{4T}|M_T(f+f_c) + M_T(f-f_c)|^2
    \end{equation}
    $M(f+f_c)$ and $M(f-f_c)$ have zero overlap as $T\rightarrow\infty$ if $f_c$ is larger than the 
    bandwidth of $M(f)$, so
    \begin{equation}
        S_{\varphi}(f) = \frac{1}{4}S_M(f+f_c) + \frac{1}{4}S_M(f-f_c)
    \end{equation}

    \begin{figure}[h]
        \centering
        \includegraphics[width=5cm, height=7cm]{psdaskfskpsk}
        \caption{PSD of ASK (on-off), PSK and FSK (polar)}
    \end{figure}

    For FSK,
    \begin{equation}
        \varphi_{FSK}(t) = \sum_k a_kp(t-kT_b)\cos\omega_{c_1}t + \sum_k(1-a_k)p(t-kT_b)\cos\omega_{c_0}t
    \end{equation}
    where baseband signal $m_1 = \sum_k a_kp(t-kT_b)$ and $m_0 = \sum_k (1-a_k)p(t-kT_b)$ the PSD would be
    \begin{equation}
        S_{FSK}(f) = \frac{1}{4}S_{M_0}(f+f_c) + \frac{1}{4}S_{M_0}(f-f_c)
            + \frac{1}{4}S_{M_1}(f+f_c) + \frac{1}{4}S_{M_1}(f-f_c)
    \end{equation}

    \subsection{Demodulation}
    \subsubsection{ASK Detection}
    ASK can be demodulated both coherently (using synchronous detection) or non-coherently with an envelope 
    detector. The coherent method takes more processing but has better performance, especially when the 
    SNR is low. When the SNR is high, envelope detection works just as well and is simpler, so coherent
    demodulation is not used often. 
    \subsubsection{FSK Detection}
    Since binary FSK can be viewed as two ASK signals ($a_k, b_k$) and carriers, $f_{c_0}$ and $f_{c_1}$, it
    can be demodulated coherently or non-coherently. In non-coherent detection, the incoming signal is
    applied to a pair of narrowband filters, $H_0(f)$ and $H_1(f)$, tuned to $f_{c_0}$ and $f_{c_1}$. Each 
    filter is followed by an envelope detector. The two methods are shown in the next figure. 
    
    \begin{figure}[h]
        \centering
        \includegraphics[width=13cm, height=7cm]{fskdemod}
        \caption{Non-coherent and Coherent Demodulation for FSK}
    \end{figure}

    \subsubsection{PSK Detection}
    In binary PSK, a $1$ is transmitted by a pulse $A\cos \omega_ct$ and a $0$ is transmitted with a 
    pulse $A\cos(\omega_ct + \pi) = -A\cos\omega_ct$. The information resides in the phase, as such these
    signals cannot be demodulated with an envelope detector. The demodulator is shown in the next
    figure. 

    \begin{figure}[h]
        \centering
        \includegraphics[width=5cm, height=3cm]{pskdemod}
        \caption{Coherent Demodulation for PSK}
    \end{figure}

    \subsubsection{Differential PSK}
    DPSK, differential PSK uses differential detection, where the receiver detects the relative phase change
    between successively modulated phases: $\theta_k$ and $\theta_{k-1}$. Since the phase in binary PSK is
    $0$ or $\pi$, the transmitter can encode the information into the phase difference $\theta_k - \theta_{k-1}$.
    For example a phase difference of $0$ would represent $0$b and a phase difference of $\pi$ would represent
    $1$b. This is known as differential encoding. 
    
    \begin{figure}[h]
        \centering
        \includegraphics[width=7cm, height=7cm]{dpsk}
        \caption{DPSK}
    \end{figure}
    
    In the demodulator, we avoid generation of an LO since
    the signal is the carrier with a sign ambiguity. We can delay the signal by $T_b$. If the pulse is identical
    to the previous one, the product 
    \begin{equation}
        y(t) = A^2\cos^2\omega_ct = (A^2/2)(1+\cos(2\omega_ct))
    \end{equation}
    and low pass filtered to
    \begin{equation}
        z(t) = A^2/2
    \end{equation}
    and the bit is decided to be $0$.
    On the other hand if the pulse and the previous pulse are opposite polarity, 
    \begin{equation}
        y(t) = -A^2\cos^2\omega_ct = -(A^2/2)(1+\cos(2\omega_ct))
    \end{equation}
    and low pass filtered to
    \begin{equation}
        z(t) = -A^2/2
    \end{equation}
    and the bit is decided to be $1$.


\end{document}