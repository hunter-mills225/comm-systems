\documentclass{article}
\usepackage{graphicx}
\usepackage{amsmath}
\usepackage{amsfonts}
\renewcommand{\baselinestretch}{1}
\setlength{\textheight}{9in}
\setlength{\textwidth}{6.5in}
\setlength{\headheight}{0in}
\setlength{\headsep}{0in}
\setlength{\topmargin}{0in}
\setlength{\oddsidemargin}{0in}
\setlength{\evensidemargin}{0in}
\setlength{\parindent}{.3in}
\graphicspath{{../../notes/images/}}


\title{Communication Systems Lab 2}
\author{Hunter Mills}
\date{\today}

\begin{document}
    \maketitle

    \medskip

    \section{Introduction}
    Analog modulation techniques were an important part in the communication protocols that we use today.
    Some of the more common analog modulation schemes were Double Sideband Amplitude modulation, Amplitude
    Modulation, AM Vestigial and FM. AM and FM were very common and used in radios and communication systems.
    This lab deals with modulating a message signal with DSB, AM and FM modulation techniques. 
    \section{Double Sideband}
    Double Sideband AM is a technique that modulates the message signal by multiplying it with a carrier
    signal that would shift the spectrum up to the carrier frequency. The formula for DSB modulation is
    \begin{equation}
        s_{DSB}(t) = m(t)\cos2\pi f_ct
    \end{equation}
    The message signal used for each modulation technique is shown in figure 1.
    \begin{figure}[!htb]
        \centering
        \includegraphics[width=7cm, height=5cm]{message}
        \caption{Message Signal for DSB, AM and FM}
    \end{figure}
    After multiplying the the signal the modulated signal and spectrum are shown in figure 2.
    \begin{figure}[!htb]
        \centering
        \includegraphics[width=7cm, height=5cm]{dsb_message}
        \caption{DSB Modulation}
    \end{figure}
    In this lab the demodulation method is using a coherent LO to down convert the received signal back to
    base-band. 
    \begin{equation}
        m(t) = s_{DSB}(t)\cos2\pi f_ct
    \end{equation}
    which is then low pass filtered to remove the images, Figure 3 shows the received signal and its spectrum
    \begin{figure}[!htb]
        \centering
        \includegraphics[width=7cm, height=5cm]{dsb_demod}
        \caption{DSB Demodulation}
    \end{figure}


    \section{Amplitude Modulation}

    AM is very similar to DSB but it uses a DC offset to modulate the carrier into the message.
    \begin{equation}
        s_{AM} = [A + m(t)]*\cos2\pi f_ct
    \end{equation}
    \begin{figure}[!htb]
        \centering
        \includegraphics[width=7cm, height=5cm]{am_message}
        \caption{AM Modulation}
    \end{figure}
    From figure 4, you can see that $m(t)$ was modulated along with a DC term by the carrier. In the spectrum
    you can see the peak at $f_c$ whereas in DSB there is no carrier spike since the carrier is not 
    modulated into the transmitted message. 

    For AM demodulation, there are a couple techniques that could be used. In this lab we use the non-
    coherent envelope detector. The first step is to rectify the signal so its fully positive and then 
    the rectified signal is low pass filtered to remove the carrier. 
    \begin{figure}[!htb]
        \centering
        \includegraphics[width=7cm, height=5cm]{am_demod}
        \caption{AM Modulation}
    \end{figure}

    \subsection{Discussion}
    \subsubsection{What is the modulation index of the AM signal?}

    The modulation index is calculated using
    \begin{equation}
        \mu = \frac{m_P}{A}
    \end{equation}
    where $m_p = 1$ and $A = 1$ so the modulation index $\mu = 1$.

    \subsubsection{Could you set the modulation index to .5 and demodulate the signal? How about
    1.5? Why or why not?}

    For an AM signal to be demodulated, the modulation index must be $0 \le \mu \le 1$. So you would
    be able to demodulate the signal when $\mu = 0.5$ but not when $\mu = 1.5$. This is because 
    the envelope would cross $y = 0$ so the envelope would be incorrectly demodulated. 

    \subsubsection{What is the difference between the envelope detector and coherent demodulation? 
    What would happen is you demodulated the DSB signal with an envelope detector?}

    The envelope detector consists of a rectifier which is followed by a low pass filter to find the 
    envelope of the signal. Whereas the coherent demodulator consists of a mixer with the carrier 
    frequency and then a low pass filter. The coherent demodulator needs to have a coherent 
    LO for the mixer. If a DSB signal was demodulated with a envelope detector then, since the DSB
    signal is centered around 0 (no DC offset), the envelope would not correlate to the sent signal.

    \section{Frequency Modulation}
    In this section of the lab, the same message signal is used that was shown in figure 1. The FM 
    modulated signal (time and frequency domains) are shown in figure 6. FM signals are modulated using
    \begin{equation}
        \varphi_{FM}(t) = A\cos [\omega_ct + k_f\int_{-\infty}^{t}m(\alpha)d\alpha]
    \end{equation}

    \begin{figure}[!htb]
        \centering
        \includegraphics[width=7cm, height=5cm]{fm_mess}
        \caption{FM Modulation}
    \end{figure}

    The Phase modulated signal is shown in figure 7. PM signals are modulated with
    \begin{equation}
        \varphi_{PM}(t) = A\cos[\omega_ct+k_pm(t)]
    \end{equation}

    \begin{figure}[!htb]
        \centering
        \includegraphics[width=7cm, height=5cm]{fm_mess}
        \caption{FM Modulation}
    \end{figure}

    The FM signal is demodulated by taking the derivative of the received signal and then passing
    that through an envelope detector. The demodulated message is shown in figure 8.

    \begin{figure}[!htb]
        \centering
        \includegraphics[width=7cm, height=5cm]{fm_demon}
        \caption{FM Modulation}
    \end{figure}

    \subsection{Discussion}
    \subsubsection{What is the bandwidth of the FM and PM signals? How does that compare to the
    theoretical bandwidth of $B_y = 2(\delta F + B_x)$ where $B_x = 25Hz$?}

    The bandwidth of the FM and PM signals seem to be approximately 200Hz. The frequency deviation
    $\Delta F = k_f \frac{m_{max}-m_{min}}{4\pi} = 160\pi / 2\pi = 80$. The theoretical bandwidth is
    $185$Hz which is fairly close to the measured bandwidth. 

    \subsubsection{Describe the FM demodulator.}

    The FM demodulator first takes the derivative of the received signal. This pulls out the message 
    content from the frequency. This is then passed into a envelope detector where the it is rectified 
    and low pass filtered to demodulate the message. 

    \section{Try it Yourself}
    \subsection{Increase $k_f$ by a factor of $2$}
    \subsubsection{What happens to the modulated signal in the time domain?}
    In the time domain the change in the frequency is more noticeable since the frequency deviation doubled
    from the previous value.
    \subsubsection{What happens to the signal spectrum?}
    The signal spectrum also doubles in width, with the sidelobes being centered around $100$Hz from the
    center frequency. The new bandwidth is about $400$Hz.
    \subsection{Can you demodulate this signal with the same demodulator?}
    No this signal is no longer able to be demodulated with this receiver. The most negative part of the 
    rectified signal (also the lowest frequency) shows that the signal is no longer the same shape as the 
    transmitted message. One way I found to fix this was to use the abs function to rectify the signal instead
    of only taking the positive values as in the matlab script. 

    \section{Conclusion}
    In conclusion, this lab has shown the main aspects of the modulator and demodulator for DSB, AM and 
    FM. These analog communication standards were widely used and are a building block for more complex
    communication protocols. 

    \section{Appendix}
    \subsection{DSB}
        \begin{verbatim}
            %!----------------------------
            %! @file      DSBdemfilt.m
            %!----------------------------
            
            %! Adding path to functions
            addpath(genpath('functions'));
            
            %! Script Variables
            ts      = 1e-4;
            fs      = 1/ts;
            fc      = 300;
            t       = -.04 : ts : .04;
            Lfft    = length(t); Lfft = 2^ceil(log2(Lfft) + 1);
            BW_m    = 150; % Bandwidth of the triangle function in Hz
            f       = (-Lfft/2 : Lfft/2-1) * fs / Lfft;
            
            %-----------------
            %! Message
            %-----------------
            m_sig  = triangle_signal((t+.01)/.01) - triangle_signal((t-.01)/.01);
            M_fre = fftshift(fft(m_sig, Lfft));
            
            %-----------------
            %! Modulation
            %-----------------
            s_dsb = m_sig .* cos(2*pi*fc*t);
            S_dsb = fftshift(fft(s_dsb, Lfft));
            
            %-----------------
            %! Demodulation
            %-----------------
            % Multiply with carrier
            s_dem = 2 * s_dsb .* cos(2*pi*fc*t);
            S_dem = fftshift(fft(s_dem, Lfft));
            
            % Low pass filter
            h     = fir1(40, BW_m / fs);
            s_rec = filter(h, 1, s_dem);
            S_rec = fftshift(fft(s_rec, Lfft));
            
            %-----------------
            %! Plot
            %-----------------
            Trange = [-.025 .025 -2 2]; % Time domain range for plots
            Frange = [-700 700 0 200];  % Frequency domain range for plots
            
            % 1.1 Plot baseband signal in time and freq domain
            figure(1)
            subplot(2,1,1)
            plot(t, m_sig)
            title('m(t)'); ylabel('Amp'); xlabel('Time (s)'); axis(Trange);
            subplot(2,1,2)
            plot(f, abs(M_fre))
            title('M(f)'); ylabel('Amp'); xlabel('Frequency (Hz)'); axis(Frange);
            
            % 1.2 Plot AM signal in t and f
            figure(2)
            subplot(2,1,1)
            plot(t, s_dsb)
            title('s(t)'); ylabel('Amp'); xlabel('Time (s)'); axis(Trange);
            subplot(2,1,2)
            plot(f, abs(S_dsb))
            title('S(f)'); ylabel('Amp'); xlabel('Frequency (Hz)'); axis(Frange);
            
            % 1.3 Plot Rx signal in freq domain, filtered Rx signal in freq and time
            figure(3)
            subplot(3,1,1)
            plot(f, abs(S_dem))
            title('S_{dem}(f) Pre-filter'); ylabel('Amp'); xlabel('Freq (Hz)'); axis(Frange);
            subplot(3,1,2)
            plot(f, abs(S_rec))
            title('S_{rec}(f) Post-filt'); ylabel('Amp'); xlabel('Frequency (Hz)'); axis(Frange);
            subplot(3,1,3)
            plot(t, s_rec)
            title('s_{rec}(t) Post-filt'); ylabel('Amp'); xlabel('Time (s)'); axis(Trange);
        \end{verbatim}

    \subsection{AM}
        \begin{verbatim}
            %!---------------------------------
            %! @file      AMdemfilt.m
            %!---------------------------------
            
            %! Adding path to functions
            addpath(genpath('functions'));
            
            %! Script Variables
            ts    = 1e-4;
            fs    = 1/ts;
            fc    = 500;
            t     = -.04 : ts : .04;
            Lfft  = length(t); Lfft = 2^ceil(log2(Lfft) + 1);
            BW_m  = 150; % Bandwidth of the triangle function in Hz
            f     = (-Lfft/2 : Lfft/2-1) * fs / Lfft;
            
            %-----------------
            %! Message
            %-----------------
            m_sig = triangle_signal((t+.01)/.01) - triangle_signal((t-.01)/.01);
            M_sig = fftshift(fft(m_sig, Lfft));
            
            %-----------------
            %! Modulation
            %-----------------
            % AM signal is created by adding DC offset to DSB (DC causes carrier in the Tx signal)
            s_am  = (1 + m_sig) .* cos(2*pi*fc*t);
            S_am  = fftshift(fft(s_am, Lfft));
            
            %-----------------
            %! Demodulation
            %-----------------
            % Demodulation starts with rectifier
            s_dem = s_am .* (s_am > 0);
            S_dem = fftshift(fft(s_dem, Lfft));
            
            % Low pass filter
            h     = fir1(40, BW_m*ts);
            s_rec = filter(h, 1, s_dem);
            S_rec = fftshift(fft(s_rec, Lfft));
            
            %-----------------
            %! Plot
            %-----------------
            Trange = [-.025 .025 -2 2]; % Time domain range for plots
            Frange = [-700 700 0 200];  % Frequency domain range for plots
            
            % 1.4 AM signal in  time and freq
            figure(1)
            subplot(2,1,1)
            plot(t, s_am)
            title('s_{am}(t)'); ylabel('Amp'); xlabel('Time (s)'); axis(Trange);
            subplot(2,1,2)
            plot(f, abs(S_am))
            title('S_{am}(f)'); ylabel('Amp'); xlabel('Frequency (Hz)'); axis(Frange);
            
            % 1.5 Rectifed and Demodulated signal in time
            figure(2)
            subplot(2,1,1)
            plot(t, s_dem)
            title('s_{dem}(t)'); ylabel('Amp'); xlabel('Time (s)'); axis(Trange);
            subplot(2,1,2)
            plot(t, s_rec)
            title('s_{rec}(t)'); ylabel('Amp'); xlabel('Time (s)'); axis([-.025 .025 -.5 1]);
        \end{verbatim}

    \section{FM}
        \begin{verbatim}
            %!--------------------------------
            %! @file      FM.m
            %!--------------------------------
            
            %! Adding path to functions
            addpath(genpath('functions'));
            
            %! Script Variables
            ts    = 1e-4;
            fs    = 1/ts;
            fc    = 300;
            t     = -.04 : ts : .04;
            Lfft  = length(t); Lfft = 2^ceil(log2(Lfft) + 1);
            BW_m  = 100; % Bandwidth of the triangle function in Hz
            f     = (-Lfft/2 : Lfft/2-1) * fs / Lfft;
            
            %-----------------
            %! Message
            %-----------------
            m_sig = triangle_signal((t+.01)/.01) - triangle_signal((t-.01)/.01);
            M_sig = fftshift(fft(m_sig, Lfft));
            
            %-----------------
            %! Modulation
            %-----------------
            kf      = 160 * pi;
            m_intg  = kf * ts * cumsum(m_sig);
            s_fm    = cos(2*pi*fc*t + m_intg);
            s_pm    = cos(2*pi*fc*t + pi*m_sig);
            S_fm    = fftshift(fft(s_fm, Lfft));
            S_pm    = fftshift(fft(s_pm, Lfft));
            
            %-----------------------
            %! FM Demodulation
            %-----------------------
            s_fmdem = diff([s_fm(1) s_fm]) * fs / kf;   % Not sure why we diff sample 1 with its self
            S_fmdem = fftshift(fft(s_fmdem, Lfft));
            s_fmrec = s_fmdem .* (s_fmdem > 0);         % Rectifier
            S_fmrec = fftshift(fft(s_fmrec, Lfft));
            
            % Low pass filter
            h       = fir1(80, BW_m*ts);
            s_dec   = filter(h, 1, s_fmrec);
            
            %-----------------
            %! Plots
            %-----------------
            Frange = [-700 700 0 200];  % Frequency domain range for plots
            
            % 2.1 FM signal in time and frequency
            figure(1)
            subplot(2,1,1)
            plot(t, s_fm)
            title('s_{fm}(t)'); ylabel('Amp'); xlabel('Time (s)');
            subplot(2,1,2)
            plot(f, abs(S_fm))
            title('S_{fm}(f)'); ylabel('Amp'); xlabel('Frequency (Hz)'); axis(Frange);
            
            % 2.2 PM signal in time and frequency
            figure(2)
            subplot(2,1,1)
            plot(t, s_pm)
            title('s_{pm}(t)'); ylabel('Amp'); xlabel('Time (s)');
            subplot(2,1,2)
            plot(f, abs(S_pm))
            title('S_{pm}(f)'); ylabel('Amp'); xlabel('Frequency (Hz)'); axis(Frange);
            
            % Test
            figure(3)
            subplot(2,1,1)
            plot(t, s_fmdem)
            title('s_{fmdem}(t)'); ylabel('Amp'); xlabel('Time (s)');
            subplot(2,1,2)
            plot(f, abs(S_fmdem))
            title('S_{fmdem}(f)'); ylabel('Amp'); xlabel('Frequency (Hz)'); axis(Frange);
            figure(4)
            subplot(2,1,1)
            plot(t, s_fmrec)
            title('s_{fmrec}(t)'); ylabel('Amp'); xlabel('Time (s)');
            subplot(2,1,2)
            plot(f, abs(S_fmrec))
            title('S_{fmrec}(f)'); ylabel('Amp'); xlabel('Frequency (Hz)'); axis(Frange);
            
            % 2.3 Demodulated FM in time
            figure(5)
            subplot(2,1,1)
            plot(t, m_sig)
            title('s_{dec}_(t)'); ylabel('Amp'); xlabel('Time (s)');
            subplot(2,1,2)
            plot(t, s_dec)
            title('s_{dec}(t)'); ylabel('Amp'); xlabel('Time (s)');
        \end{verbatim}

\end{document}